\documentclass{beamer}

\input{preamble.tex}

\subtitle{Presentations with \protect\bftt{beamer}}

\newcommand{\alice}[1]{\todo[color=green!40]{#1}}
\newcommand{\bob}[1]{\todo[color=purple!40]{#1}}

\begin{document}

%%%%%%%%%%%%%%%%%%%%%%%%%%%%%%%%%%%%%%%%%%%%%%%%%%%%%%%%%%%%%%%%%%%%%%%%%%%%%%%
%%%%%%%%%%%%%%%%%%%%%%%%%%%%%%%%%%%%%%%%%%%%%%%%%%%%%%%%%%%%%%%%%%%%%%%%%%%%%%%
%%%%%%%%%%%%%%%%%%%%%%%%%%%%%%%%%%%%%%%%%%%%%%%%%%%%%%%%%%%%%%%%%%%%%%%%%%%%%%%
\begin{frame}
\titlepage
\end{frame}

%%%%%%%%%%%%%%%%%%%%%%%%%%%%%%%%%%%%%%%%%%%%%%%%%%%%%%%%%%%%%%%%%%%%%%%%%%%%%%%
%%%%%%%%%%%%%%%%%%%%%%%%%%%%%%%%%%%%%%%%%%%%%%%%%%%%%%%%%%%%%%%%%%%%%%%%%%%%%%%
%%%%%%%%%%%%%%%%%%%%%%%%%%%%%%%%%%%%%%%%%%%%%%%%%%%%%%%%%%%%%%%%%%%%%%%%%%%%%%%
\section{Presentations with \protect\bftt{beamer}}

%%%%%%%%%%%%%%%%%%%%%%%%%%%%%%%%%%%%%%%%%%%%%%%%%%%%%%%%%%%%%%%%%%%%%%%%%%%%%%%
%%%%%%%%%%%%%%%%%%%%%%%%%%%%%%%%%%%%%%%%%%%%%%%%%%%%%%%%%%%%%%%%%%%%%%%%%%%%%%%
%%%%%%%%%%%%%%%%%%%%%%%%%%%%%%%%%%%%%%%%%%%%%%%%%%%%%%%%%%%%%%%%%%%%%%%%%%%%%%%
\begin{frame}[fragile]{\insertsection}
\begin{itemize}
\item Beamer is a package for creating presentations (such as this one!) in
\LaTeX{}.
\item It provides the \bftt{beamer} document class.
\item Use the \bftt{frame} environment to create slides.
\end{itemize}
\begin{minipage}{0.55\linewidth}
\inputminted[fontsize=\scriptsize,frame=single,resetmargins]{latex}%
  {beamer-minimal.tex}
\end{minipage}
\begin{minipage}{0.35\linewidth}
% trim: l b r t
\includegraphics[width=\textwidth,clip,trim=1in 1in 1in 1in]{beamer-minimal.pdf}
\end{minipage}
\end{frame}

%%%%%%%%%%%%%%%%%%%%%%%%%%%%%%%%%%%%%%%%%%%%%%%%%%%%%%%%%%%%%%%%%%%%%%%%%%%%%%%
%%%%%%%%%%%%%%%%%%%%%%%%%%%%%%%%%%%%%%%%%%%%%%%%%%%%%%%%%%%%%%%%%%%%%%%%%%%%%%%
%%%%%%%%%%%%%%%%%%%%%%%%%%%%%%%%%%%%%%%%%%%%%%%%%%%%%%%%%%%%%%%%%%%%%%%%%%%%%%%
\begin{frame}[fragile]{\insertsection: Following Along}

\begin{itemize}
\item As we go through the following slides, try out the examples by typing them
into the example document on \wllogo.
\end{itemize}
\vskip 2ex
\begin{center}
\fbox{\href{\wlnewdoc{beamer-minimal.tex}}{%
Click to open the example document in \wllogo{}}}
\end{center}
\end{frame}

%%%%%%%%%%%%%%%%%%%%%%%%%%%%%%%%%%%%%%%%%%%%%%%%%%%%%%%%%%%%%%%%%%%%%%%%%%%%%%%
%%%%%%%%%%%%%%%%%%%%%%%%%%%%%%%%%%%%%%%%%%%%%%%%%%%%%%%%%%%%%%%%%%%%%%%%%%%%%%%
%%%%%%%%%%%%%%%%%%%%%%%%%%%%%%%%%%%%%%%%%%%%%%%%%%%%%%%%%%%%%%%%%%%%%%%%%%%%%%%
\begin{frame}[fragile]
\frametitle{\insertsection: Frames}
\begin{itemize}
\item Use \cmdbs{frametitle} to give the frame a title.
\item Then add content to the frame.
\item The source for this frame looks like:
\vskip 2ex
\inputminted[fontsize=\scriptsize,frame=single,resetmargins]{latex}%
  {beamer-frame.tex}
\end{itemize}
\end{frame}

%%%%%%%%%%%%%%%%%%%%%%%%%%%%%%%%%%%%%%%%%%%%%%%%%%%%%%%%%%%%%%%%%%%%%%%%%%%%%%%
%%%%%%%%%%%%%%%%%%%%%%%%%%%%%%%%%%%%%%%%%%%%%%%%%%%%%%%%%%%%%%%%%%%%%%%%%%%%%%%
%%%%%%%%%%%%%%%%%%%%%%%%%%%%%%%%%%%%%%%%%%%%%%%%%%%%%%%%%%%%%%%%%%%%%%%%%%%%%%%
\begin{frame}[fragile]{\insertsection: Sections}
\begin{itemize}
\item You can use \cmdbs{section}s to group your \bftt{frame}s, and
\bftt{beamer} will use them to create an automatic outline.
\item To generate an outline, use the \cmdbs{tableofcontents} command. Here's
one for this presentation. The \bftt{currentsection} option highlights the current section.
\vskip 2ex
\begin{exampletwouptiny}
\tableofcontents[currentsection]
\end{exampletwouptiny}
\end{itemize}
\end{frame}

%%%%%%%%%%%%%%%%%%%%%%%%%%%%%%%%%%%%%%%%%%%%%%%%%%%%%%%%%%%%%%%%%%%%%%%%%%%%%%%
%%%%%%%%%%%%%%%%%%%%%%%%%%%%%%%%%%%%%%%%%%%%%%%%%%%%%%%%%%%%%%%%%%%%%%%%%%%%%%%
%%%%%%%%%%%%%%%%%%%%%%%%%%%%%%%%%%%%%%%%%%%%%%%%%%%%%%%%%%%%%%%%%%%%%%%%%%%%%%%
\begin{frame}[fragile]{\insertsection: Multiple Columns}
\begin{columns}
\begin{column}{0.4\textwidth}
\begin{itemize}
\item Use the \bftt{columns} and \bftt{column} environments to break the slide
into columns.
\item The argument for each \bftt{column} determines its width.
\item See also the \bftt{multicol} package, which automatically breaks your
content into columns.
\end{itemize}
\end{column}
\begin{column}{0.6\textwidth}
\begin{minted}[fontsize=\scriptsize,frame=single]{latex}
\begin{columns}
  \begin{column}{0.4\textwidth}
    \begin{itemize}
    \item Use the columns ...
    \item The argument ...
    \item See also the ...
    \end{itemize}
  \end{column}
  \begin{column}{0.6\textwidth}
    % second column
  \end{column}
\end{columns}
\end{minted}
\end{column}
\end{columns}
\end{frame}

%%%%%%%%%%%%%%%%%%%%%%%%%%%%%%%%%%%%%%%%%%%%%%%%%%%%%%%%%%%%%%%%%%%%%%%%%%%%%%%
%%%%%%%%%%%%%%%%%%%%%%%%%%%%%%%%%%%%%%%%%%%%%%%%%%%%%%%%%%%%%%%%%%%%%%%%%%%%%%%
%%%%%%%%%%%%%%%%%%%%%%%%%%%%%%%%%%%%%%%%%%%%%%%%%%%%%%%%%%%%%%%%%%%%%%%%%%%%%%%
\begin{frame}[fragile]{\insertsection: Blocks}
\begin{itemize}
\item A \bftt{block} environment makes a titled box.
\begin{exampletwouptiny}
\begin{block}{Interesting Fact}
This is important.
\end{block}

\begin{alertblock}{Cautionary Tale}
This is really important!
\end{alertblock}
\end{exampletwouptiny}

\item How exactly they look depends on the theme\ldots
\end{itemize}
\end{frame}

%%%%%%%%%%%%%%%%%%%%%%%%%%%%%%%%%%%%%%%%%%%%%%%%%%%%%%%%%%%%%%%%%%%%%%%%%%%%%%%
%%%%%%%%%%%%%%%%%%%%%%%%%%%%%%%%%%%%%%%%%%%%%%%%%%%%%%%%%%%%%%%%%%%%%%%%%%%%%%%
%%%%%%%%%%%%%%%%%%%%%%%%%%%%%%%%%%%%%%%%%%%%%%%%%%%%%%%%%%%%%%%%%%%%%%%%%%%%%%%
\begin{frame}[fragile]
\frametitle{\insertsection: Themes}
\begin{itemize}
\item Customise the look of your presentation using themes.
\item See \url{http://deic.uab.es/~iblanes/beamer_gallery/index_by_theme.html}
for a large collection of themes.
\end{itemize}
\begin{minipage}{0.55\linewidth}
\inputminted[fontsize=\scriptsize,frame=single,resetmargins]{latex}%
  {beamer-theme.tex}
\end{minipage}
\begin{minipage}{0.35\linewidth}
% trim: l b r t
\includegraphics[width=\textwidth]{beamer-theme.pdf}
\end{minipage}
\end{frame}

%%%%%%%%%%%%%%%%%%%%%%%%%%%%%%%%%%%%%%%%%%%%%%%%%%%%%%%%%%%%%%%%%%%%%%%%%%%%%%%
%%%%%%%%%%%%%%%%%%%%%%%%%%%%%%%%%%%%%%%%%%%%%%%%%%%%%%%%%%%%%%%%%%%%%%%%%%%%%%%
%%%%%%%%%%%%%%%%%%%%%%%%%%%%%%%%%%%%%%%%%%%%%%%%%%%%%%%%%%%%%%%%%%%%%%%%%%%%%%%
\begin{frame}[fragile]{\insertsection: Animation}
\begin{itemize}
\item A frame can generate multiple slides.
\item Use the \cmdbs{pause} command to show only part of a slide.
\vskip 2ex
\begin{exampletwouptinynoframe}
\begin{itemize}
\item Can you feel the 
\pause \item anticipation?
\end{itemize}
\end{exampletwouptinynoframe}
\vskip 2ex
\item There many more clever ways of making animations in \bftt{beamer}; see
also the \cmdbs{only}, \cmdbs{alt}, and \cmdbs{uncover} commands.
\end{itemize}
\end{frame}

%%%%%%%%%%%%%%%%%%%%%%%%%%%%%%%%%%%%%%%%%%%%%%%%%%%%%%%%%%%%%%%%%%%%%%%%%%%%%%%
%%%%%%%%%%%%%%%%%%%%%%%%%%%%%%%%%%%%%%%%%%%%%%%%%%%%%%%%%%%%%%%%%%%%%%%%%%%%%%%
%%%%%%%%%%%%%%%%%%%%%%%%%%%%%%%%%%%%%%%%%%%%%%%%%%%%%%%%%%%%%%%%%%%%%%%%%%%%%%%
\begin{frame}[fragile]{\insertsection: Exercise}

Recreate Peter Norvig's excellent ``Gettysburg Powerpoint Presentation'' in \bftt{beamer}.\footnote{\url{http://norvig.com/Gettysburg}}

\begin{enumerate}
\item Open this exercise in \wllogo{}:
\begin{center}
\fbox{\href{\wlnewdoc{beamer-exercise.tex}}{%
Click to open this exercise in \wllogo{}}}
\end{center}
\vskip 2ex
\item Download this image to your computer and upload it to \wllogo{} via the
files menu.
\begin{center}
\fbox{\href{\fileuri/gettysburg_graph.png?dl=1}{Click to download image}}
\end{center}
\vskip 2ex
\item Add \LaTeX{} commands to the text to make it look like this one:
\begin{center}
\fbox{\href{\fileuri/beamer-exercise-solution.pdf}{%
Click to open the model document}}
\end{center}
\end{enumerate}
\end{frame}

\end{document}
