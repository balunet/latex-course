\documentclass{beamer}

\input{preamble.tex}

%\subtitle{Basics and writing documents}

\begin{document}

%%%%%%%%%%%%%%%%%%%%%%%%%%%%%%%%%%%%%%%%%%%%%%%%%%%%%%%%%%%%%%%%%%%%%%%%%%%%%%%
%%%%%%%%%%%%%%%%%%%%%%%%%%%%%%%%%%%%%%%%%%%%%%%%%%%%%%%%%%%%%%%%%%%%%%%%%%%%%%%
%%%%%%%%%%%%%%%%%%%%%%%%%%%%%%%%%%%%%%%%%%%%%%%%%%%%%%%%%%%%%%%%%%%%%%%%%%%%%%%
\begin{frame}
\titlepage
\end{frame}

\begin{frame}{Outline}
\tableofcontents
\end{frame}

%%%%%%%%%%%%%%%%%%%%%%%%%%%%%%%%%%%%%%%%%%%%%%%%%%%%%%%%%%%%%%%%%%%%%%%%%%%%%%%
%%%%%%%%%%%%%%%%%%%%%%%%%%%%%%%%%%%%%%%%%%%%%%%%%%%%%%%%%%%%%%%%%%%%%%%%%%%%%%%
%%%%%%%%%%%%%%%%%%%%%%%%%%%%%%%%%%%%%%%%%%%%%%%%%%%%%%%%%%%%%%%%%%%%%%%%%%%%%%%

\section{Introduction}
\begin{frame}{Why \LaTeX{}?}
\begin{itemize}
\item It makes beautiful documents
\begin{itemize}
\item Especially mathematics
\end{itemize}
%
\item It makes you focus only on the content
\begin{itemize}
\item No need to worry about how the content is displayed.
\end{itemize}
%
\item It is powerful --- you can extend it
\begin{itemize}
\item Packages for papers, presentations, spreadsheets, \ldots
\end{itemize}
\item It is widely used in academic and scientific scopes.
\end{itemize}
\end{frame}

%%%%%%%%%%%%%%%%%%%%%%%%%%%%%%%%%%%%%%%%%%%%%%%%%%%%%%%%%%%%%%%%%%%%%%%%%%%%%%%
%%%%%%%%%%%%%%%%%%%%%%%%%%%%%%%%%%%%%%%%%%%%%%%%%%%%%%%%%%%%%%%%%%%%%%%%%%%%%%%
%%%%%%%%%%%%%%%%%%%%%%%%%%%%%%%%%%%%%%%%%%%%%%%%%%%%%%%%%%%%%%%%%%%%%%%%%%%%%%%
\begin{frame}[fragile]{How does it work?}
\begin{itemize}
\item You write your document in \texttt{plain text} with \cmd{commands} that
describe its structure and meaning.
\item The \texttt{latex} program processes your text and commands to produce a
beautifully formatted document.
\end{itemize}
\vskip 2ex
\begin{center}
\begin{minted}[frame=single]{latex}
Int. Business' students are \emph{pretty} cool.
\end{minted}
\vskip 2ex
\tikz\node[single arrow,fill=gray,font=\ttfamily\bfseries,%
  rotate=270,xshift=-1em]{latex};
\vskip 2ex
\fbox{Int. Business' students      are \emph{pretty} cool.}
\end{center}
\end{frame}

%%%%%%%%%%%%%%%%%%%%%%%%%%%%%%%%%%%%%%%%%%%%%%%%%%%%%%%%%%%%%%%%%%%%%%%%%%%%%%%
%%%%%%%%%%%%%%%%%%%%%%%%%%%%%%%%%%%%%%%%%%%%%%%%%%%%%%%%%%%%%%%%%%%%%%%%%%%%%%%
%%%%%%%%%%%%%%%%%%%%%%%%%%%%%%%%%%%%%%%%%%%%%%%%%%%%%%%%%%%%%%%%%%%%%%%%%%%%%%%
\begin{frame}[fragile]{More examples of commands and their output\ldots}
\begin{exampletwoup}
\begin{itemize}
\item Rice
\item Rabbit
\item Chicken
\end{itemize}
\end{exampletwoup}
\vskip 2ex
\begin{exampletwoup}
\begin{figure}
\includegraphics{gerbil}
\end{figure}
\end{exampletwoup}
\vskip 2ex
\begin{exampletwoup}
\begin{equation}
\alpha + \beta + 1
\end{equation}
\end{exampletwoup}

\end{frame}

%%%%%%%%%%%%%%%%%%%%%%%%%%%%%%%%%%%%%%%%%%%%%%%%%%%%%%%%%%%%%%%%%%%%%%%%%%%%%%%
%%%%%%%%%%%%%%%%%%%%%%%%%%%%%%%%%%%%%%%%%%%%%%%%%%%%%%%%%%%%%%%%%%%%%%%%%%%%%%%
%%%%%%%%%%%%%%%%%%%%%%%%%%%%%%%%%%%%%%%%%%%%%%%%%%%%%%%%%%%%%%%%%%%%%%%%%%%%%%%
\begin{frame}[fragile]{Change your mind!}

\begin{itemize}
\item Use commands to describe `what it is', not `how it looks'.
\item Focus on your content.
\item Let \LaTeX{} do its job.
\end{itemize}
\end{frame}

%%%%%%%%%%%%%%%%%%%%%%%%%%%%%%%%%%%%%%%%%%%%%%%%%%%%%%%%%%%%%%%%%%%%%%%%%%%%%%%
%%%%%%%%%%%%%%%%%%%%%%%%%%%%%%%%%%%%%%%%%%%%%%%%%%%%%%%%%%%%%%%%%%%%%%%%%%%%%%%
%%%%%%%%%%%%%%%%%%%%%%%%%%%%%%%%%%%%%%%%%%%%%%%%%%%%%%%%%%%%%%%%%%%%%%%%%%%%%%%
\section{The Basics}

%%%%%%%%%%%%%%%%%%%%%%%%%%%%%%%%%%%%%%%%%%%%%%%%%%%%%%%%%%%%%%%%%%%%%%%%%%%%%%%
%%%%%%%%%%%%%%%%%%%%%%%%%%%%%%%%%%%%%%%%%%%%%%%%%%%%%%%%%%%%%%%%%%%%%%%%%%%%%%%
%%%%%%%%%%%%%%%%%%%%%%%%%%%%%%%%%%%%%%%%%%%%%%%%%%%%%%%%%%%%%%%%%%%%%%%%%%%%%%%
\begin{frame}{Outline}
\tableofcontents[currentsection]
\end{frame}


%%%%%%%%%%%%%%%%%%%%%%%%%%%%%%%%%%%%%%%%%%%%%%%%%%%%%%%%%%%%%%%%%%%%%%%%%%%%%%%
%%%%%%%%%%%%%%%%%%%%%%%%%%%%%%%%%%%%%%%%%%%%%%%%%%%%%%%%%%%%%%%%%%%%%%%%%%%%%%%
%%%%%%%%%%%%%%%%%%%%%%%%%%%%%%%%%%%%%%%%%%%%%%%%%%%%%%%%%%%%%%%%%%%%%%%%%%%%%%%
\subsection*{Getting started}
\begin{frame}[fragile]{\insertsubsection}
\begin{itemize}
\item A minimal \LaTeX{} document:
\inputminted[frame=single]{latex}{basics.tex}
\item Commands start with a \emph{backslash} \keystrokebftt{\bs}.
\item Every document starts with a \cmdbs{documentclass} command.
\item The \emph{argument} in curly braces \keystrokebftt{\{} \keystrokebftt{\}} tells \LaTeX{} what kind of document we are creating: an \bftt{article}.
\item A percent sign \keystrokebftt{\%} starts a \emph{comment} --- \LaTeX{}
will ignore the rest of the line.
\end{itemize}
\end{frame}

%%%%%%%%%%%%%%%%%%%%%%%%%%%%%%%%%%%%%%%%%%%%%%%%%%%%%%%%%%%%%%%%%%%%%%%%%%%%%%%
%%%%%%%%%%%%%%%%%%%%%%%%%%%%%%%%%%%%%%%%%%%%%%%%%%%%%%%%%%%%%%%%%%%%%%%%%%%%%%%
%%%%%%%%%%%%%%%%%%%%%%%%%%%%%%%%%%%%%%%%%%%%%%%%%%%%%%%%%%%%%%%%%%%%%%%%%%%%%%%
\begin{frame}[fragile]{\insertsubsection{} with \wllogo}
\begin{itemize}
\item Overleaf is a website for writing documents in \LaTeX.
\item It `compiles' your \LaTeX{} automatically to show you the results.
\vskip 2em
\begin{center}
\fbox{\href{\wlnewdoc{basics.tex}}{%
Click here to open an example document in \wllogo{}}}
\end{center}
\vskip 2ex
\item Use this sample document to test all commands we'll learn.
\end{itemize}
\end{frame}

%%%%%%%%%%%%%%%%%%%%%%%%%%%%%%%%%%%%%%%%%%%%%%%%%%%%%%%%%%%%%%%%%%%%%%%%%%%%%%%
%%%%%%%%%%%%%%%%%%%%%%%%%%%%%%%%%%%%%%%%%%%%%%%%%%%%%%%%%%%%%%%%%%%%%%%%%%%%%%%
%%%%%%%%%%%%%%%%%%%%%%%%%%%%%%%%%%%%%%%%%%%%%%%%%%%%%%%%%%%%%%%%%%%%%%%%%%%%%%%
\subsection*{Writing Text}
\begin{frame}[fragile]{\insertsubsection{}}
\small
\begin{itemize}
\item Type your text between \cmdbegin{document} and \cmdend{document}.
\item For the most part, you can just type your text normally.
\begin{exampletwouptiny}
Words are separated by one or more
spaces.




Paragraphs are separated by one
or more blank lines.
\end{exampletwouptiny}
\item Space in the source file is collapsed in the output.
\begin{exampletwouptiny}
The   rain       in Spain
falls mainly on the plain.
\end{exampletwouptiny}
\end{itemize}
\end{frame}

%%%%%%%%%%%%%%%%%%%%%%%%%%%%%%%%%%%%%%%%%%%%%%%%%%%%%%%%%%%%%%%%%%%%%%%%%%%%%%%
%%%%%%%%%%%%%%%%%%%%%%%%%%%%%%%%%%%%%%%%%%%%%%%%%%%%%%%%%%%%%%%%%%%%%%%%%%%%%%%
%%%%%%%%%%%%%%%%%%%%%%%%%%%%%%%%%%%%%%%%%%%%%%%%%%%%%%%%%%%%%%%%%%%%%%%%%%%%%%%
\begin{frame}[fragile]{\insertsubsection{}}
\small
\begin{itemize}
\item Quotation marks are a bit tricky:\\
use a backtick \keystroke{\`{}} on the left and an apostrophe \keystroke{\'{}} on the right.
\begin{exampletwouptiny}
Single quotes: `text'.

Double quotes: ``text''.
\end{exampletwouptiny}

\item Some common characters have special meanings in \LaTeX:\\[1ex]
\begin{tabular}{cl}
\keystrokebftt{\%} & percent sign              \\
\keystrokebftt{\#} & hash (pound / sharp) sign \\
\keystrokebftt{\&} & ampersand                 \\
\keystrokebftt{\$} & dollar sign               \\
\end{tabular}
\item If you just type these, you'll get an error. If you want one to appear in
the output, you have to \emph{escape} it by preceding it with a backslash.
\begin{exampletwoup}
\$\%\&\#!
\end{exampletwoup}
\end{itemize}
\end{frame}


\begin{frame}[fragile]{\insertsubsection{}}
\begin{itemize}

\item Use \cmdbs{emph} or \cmdbs{alert} to highlight:
\vskip 1ex
\begin{exampletwouptiny}
I should \emph{emphasise} that
this is an \alert{important} point.
\end{exampletwouptiny}
\vskip 1ex

\item Or specify bold face or italics:
\vskip 1ex
\begin{exampletwouptiny}
Text in \textbf{bold face}.
Text in \textit{italics}.
\end{exampletwouptiny}
\vskip 1ex

\item Or specify a color:
\vskip 1ex
\begin{exampletwouptiny}
It \textcolor{red}{stops}
and \textcolor{green}{starts}.
\end{exampletwouptiny}
\end{itemize}
\end{frame}

%%%%%%%%%%%%%%%%%%%%%%%%%%%%%%%%%%%%%%%%%%%%%%%%%%%%%%%%%%%%%%%%%%%%%%%%%%%%%%%
%%%%%%%%%%%%%%%%%%%%%%%%%%%%%%%%%%%%%%%%%%%%%%%%%%%%%%%%%%%%%%%%%%%%%%%%%%%%%%%
%%%%%%%%%%%%%%%%%%%%%%%%%%%%%%%%%%%%%%%%%%%%%%%%%%%%%%%%%%%%%%%%%%%%%%%%%%%%%%%
%\subsection*{Non-ASCII characters}
\begin{frame}[fragile]{\insertsubsection}
\begin{itemize}

\item If you write non-ASCII characters, you must scape accents:
\begin{itemize}
  \item \bftt{Castell\textbackslash\'{}\{o\}} $\to$ \bftt{Castell\'{o}}
  \item \bftt{Val\textbackslash\`{}\{e\}ncia} $\to$ \bftt{Val\`{e}ncia}
  \item \bftt{Espa\textbackslash\textasciitilde{}\{n\}a} $\to$ \bftt{Espa\~{n}a}
  \item \bftt{Biling\textbackslash"\{u\}ismo} $\to$ \bftt{Biling\"{u}ismo}
\end{itemize}

\item Better: add this to the preamble of the document:

\begin{minted}[fontsize=\small,frame=single]{latex}
\usepackage[utf8]{inputenc} 
\usepackage[T1]{fontenc}
\end{minted}

\item This way you won't need to scape accents or quotation marks.

\end{itemize}
\end{frame}




%%%%%%%%%%%%%%%%%%%%%%%%%%%%%%%%%%%%%%%%%%%%%%%%%%%%%%%%%%%%%%%%%%%%%%%%%%%%%%%
%%%%%%%%%%%%%%%%%%%%%%%%%%%%%%%%%%%%%%%%%%%%%%%%%%%%%%%%%%%%%%%%%%%%%%%%%%%%%%%
%%%%%%%%%%%%%%%%%%%%%%%%%%%%%%%%%%%%%%%%%%%%%%%%%%%%%%%%%%%%%%%%%%%%%%%%%%%%%%%
\begin{frame}[fragile]{Handling Errors}
\begin{itemize}
\item \LaTeX{} can get confused when trying to compile a document. 
\begin{itemize}
\item If it does, it stops with an error, which you must fix.
\end{itemize}
\item For example, if you misspell \cmdbs{emph} as \cmdbs{meph}, \LaTeX{} will
stop with an ``undefined control sequence'' error, because ``meph'' is not
one of the commands it knows.
\end{itemize}
\begin{block}{Advice on Errors}
\begin{enumerate}
\item Don't panic! Errors happen.
\item Fix them as soon as they arise. 
\begin{itemize}
  \item if what you just typed caused an error, you can start your debugging there.
\end{itemize}
\end{enumerate}
\end{block}
\end{frame}

%%%%%%%%%%%%%%%%%%%%%%%%%%%%%%%%%%%%%%%%%%%%%%%%%%%%%%%%%%%%%%%%%%%%%%%%%%%%%%%
%%%%%%%%%%%%%%%%%%%%%%%%%%%%%%%%%%%%%%%%%%%%%%%%%%%%%%%%%%%%%%%%%%%%%%%%%%%%%%%
%%%%%%%%%%%%%%%%%%%%%%%%%%%%%%%%%%%%%%%%%%%%%%%%%%%%%%%%%%%%%%%%%%%%%%%%%%%%%%%
\begin{frame}[fragile]{Exercise 1}

\begin{block}{Write this in \LaTeX:}
%\footnote{\url{http://en.wikipedia.org/wiki/Economy_of_the_United_States}}}
Val\`{e}ncia's economy is service-oriented, as nearly 84\% of the working population
is employed in service sector occupations. In 2009, Val\`{e}ncia was designated \emph{``the
29th fastest-improving European city''}. 

Its influence in commerce, education,
entertainment, media, fashion, science and the arts contributes to its status
as one of the world's ``Gamma''-rank global cities. The Val\`{e}ncia metropolitan
area had a GDP\$ amounting to \$52.7 billion, and \$28,141 per capita.

\end{block}
\vskip 2ex
\begin{center}
\fbox{\href{\wlnewdoc{basics-exercise-1.tex}}{%
Click to open this exercise in \wllogo{}}}
\end{center}

\begin{itemize}
\item Hint: watch out for characters with special meanings!
\item Once you've tried,
\fbox{\href{\wlnewdoc{basics-exercise-1-solution.tex}}{%
click here to see my solution}}.
\end{itemize}
\end{frame}

%%%%%%%%%%%%%%%%%%%%%%%%%%%%%%%%%%%%%%%%%%%%%%%%%%%%%%%%%%%%%%%%%%%%%%%%%%%%%%%
%%%%%%%%%%%%%%%%%%%%%%%%%%%%%%%%%%%%%%%%%%%%%%%%%%%%%%%%%%%%%%%%%%%%%%%%%%%%%%%
%%%%%%%%%%%%%%%%%%%%%%%%%%%%%%%%%%%%%%%%%%%%%%%%%%%%%%%%%%%%%%%%%%%%%%%%%%%%%%%
\subsection*{Writing Mathematics}
\begin{frame}[fragile]{\insertsubsection{}: Dollar Signs}
\begin{itemize}
\item We use dollar signs \keystrokebftt{\$} to mark mathematics in text.\\[1ex]
\begin{exampletwouptiny}
% not so good:
Let a and b be distinct positive
integers, and let c = a - b + 1.

% much better:
Let $a$ and $b$ be distinct positive
integers, and let $c = a - b + 1$.
\end{exampletwouptiny}
\item Always use dollar signs in pairs
\begin{itemize}
  \item one to begin the mathematics, and one to end it.
\end{itemize}
\item \LaTeX{} handles spacing automatically; it ignores your spaces.
\begin{exampletwouptiny}
Let $y=mx+b$ be \ldots

Let $y = m x + b$ be \ldots
\end{exampletwouptiny}
\end{itemize}
\end{frame}

%%%%%%%%%%%%%%%%%%%%%%%%%%%%%%%%%%%%%%%%%%%%%%%%%%%%%%%%%%%%%%%%%%%%%%%%%%%%%%%
%%%%%%%%%%%%%%%%%%%%%%%%%%%%%%%%%%%%%%%%%%%%%%%%%%%%%%%%%%%%%%%%%%%%%%%%%%%%%%%
%%%%%%%%%%%%%%%%%%%%%%%%%%%%%%%%%%%%%%%%%%%%%%%%%%%%%%%%%%%%%%%%%%%%%%%%%%%%%%%
\begin{frame}[fragile]{\insertsubsection{}: Notation}
\begin{itemize}
\item Use caret \keystrokebftt{\^} for superscripts and underscore \keystrokebftt{\_} for subscripts.
\begin{exampletwouptiny}
$y = c_2 x^2 + c_1 x + c_0$
\end{exampletwouptiny}
\vskip 2ex

\item Use curly braces \keystrokebftt{\{} \keystrokebftt{\}} to group
superscripts and subscripts.
\begin{exampletwouptiny}
$F_n = F_n-1 + F_n-2$     % oops!

$F_n = F_{n-1} + F_{n-2}$ % ok!
\end{exampletwouptiny}
\vskip 2ex

\item There are commands for Greek letters and common notation.
\begin{exampletwouptiny}
$\mu = A e^{Q/RT}$

$\Omega = \sum_{k=1}^{n} \omega_k$
\end{exampletwouptiny}

\end{itemize}
\end{frame}

%%%%%%%%%%%%%%%%%%%%%%%%%%%%%%%%%%%%%%%%%%%%%%%%%%%%%%%%%%%%%%%%%%%%%%%%%%%%%%%
%%%%%%%%%%%%%%%%%%%%%%%%%%%%%%%%%%%%%%%%%%%%%%%%%%%%%%%%%%%%%%%%%%%%%%%%%%%%%%%
%%%%%%%%%%%%%%%%%%%%%%%%%%%%%%%%%%%%%%%%%%%%%%%%%%%%%%%%%%%%%%%%%%%%%%%%%%%%%%%
\begin{frame}[fragile]{\insertsubsection{}: Equations}
\begin{itemize}
\item If it's big and scary, \emph{display} it on its own line using
\cmdbegin{equation} and \cmdend{equation}.\\[2ex]
\begin{exampletwouptiny}
The roots of a quadratic equation
are given by
\begin{equation}
x = \frac{-b \pm \sqrt{b^2 - 4ac}}
         {2a}
\end{equation}
where $a$, $b$ and $c$ are \ldots
\end{exampletwouptiny}

\item You should try this on-line \LaTeX{} equation editor:
\begin{itemize}
  \item \url{https://www.codecogs.com/latex/eqneditor.php}
\end{itemize}



\end{itemize}
\end{frame}

%%%%%%%%%%%%%%%%%%%%%%%%%%%%%%%%%%%%%%%%%%%%%%%%%%%%%%%%%%%%%%%%%%%%%%%%%%%%%%%%
%%%%%%%%%%%%%%%%%%%%%%%%%%%%%%%%%%%%%%%%%%%%%%%%%%%%%%%%%%%%%%%%%%%%%%%%%%%%%%%%
%%%%%%%%%%%%%%%%%%%%%%%%%%%%%%%%%%%%%%%%%%%%%%%%%%%%%%%%%%%%%%%%%%%%%%%%%%%%%%%%

\begin{frame}[fragile]{Exercise 2}

\begin{block}{Write this in \LaTeX:}
Let $X_1, X_2, \ldots, X_n$ be a sequence of independent and identically
distributed random variables with $\operatorname{E}[X_i] = \mu$ and
$\operatorname{Var}[X_i] = \sigma^2 < \infty$, and let
\begin{equation} \tag{1}
S_n = \frac{1}{n}\sum_{i}^{n} X_i
\end{equation}
denote their mean. Then as $n$ approaches infinity, the random variables
$\sqrt{n}(S_n - \mu)$ converge in distribution to a normal $N(0, \sigma^2)$.
\end{block}
\vskip 2ex
\begin{center}
\fbox{\href{\wlnewdoc{basics-exercise-2.tex}}{%
Click to open this exercise in \wllogo{}}}
\end{center}
\begin{itemize}
\item Hint: the command for $\infty$ is \cmdbs{infty}. What about $\sigma$ and $\mu$?
\item Once you've tried,
\fbox{\href{\wlnewdoc{basics-exercise-2-solution.tex}}{%
click here to see my solution}}.
\end{itemize}
\end{frame}

%%%%%%%%%%%%%%%%%%%%%%%%%%%%%%%%%%%%%%%%%%%%%%%%%%%%%%%%%%%%%%%%%%%%%%%%%%%%%%%
%%%%%%%%%%%%%%%%%%%%%%%%%%%%%%%%%%%%%%%%%%%%%%%%%%%%%%%%%%%%%%%%%%%%%%%%%%%%%%%
%%%%%%%%%%%%%%%%%%%%%%%%%%%%%%%%%%%%%%%%%%%%%%%%%%%%%%%%%%%%%%%%%%%%%%%%%%%%%%%

\subsection*{Lists}
\begin{frame}[fragile]{Lists}
\begin{itemize}
\item Use \bftt{itemize} environment for building unordered lists:

\begin{exampletwouptiny}
\begin{itemize}
\item Cats
\item Dogs
  \begin{itemize}
    \item Fox Terrier
    \item Damaltian
  \end{itemize}
\item Crocodiles
\end{itemize}
\end{exampletwouptiny}

\item For numbered lists, use \bftt{enumerate} environment.

\begin{exampletwouptiny}
\begin{enumerate}
\item Buy ingredients
  \begin{enumerate}
    \item Go to the supermarket
    \item Pick up products
    \item Pay them
  \end{enumerate}
\item Make your paella
\item Enjoy!
\end{enumerate}
\end{exampletwouptiny}

\end{itemize}

\end{frame}

\section{Structured Documents}

%%%%%%%%%%%%%%%%%%%%%%%%%%%%%%%%%%%%%%%%%%%%%%%%%%%%%%%%%%%%%%%%%%%%%%%%%%%%%%%
%%%%%%%%%%%%%%%%%%%%%%%%%%%%%%%%%%%%%%%%%%%%%%%%%%%%%%%%%%%%%%%%%%%%%%%%%%%%%%%
%%%%%%%%%%%%%%%%%%%%%%%%%%%%%%%%%%%%%%%%%%%%%%%%%%%%%%%%%%%%%%%%%%%%%%%%%%%%%%%
\begin{frame}{Outline}
\tableofcontents[currentsection]
\end{frame}


%%%%%%%%%%%%%%%%%%%%%%%%%%%%%%%%%%%%%%%%%%%%%%%%%%%%%%%%%%%%%%%%%%%%%%%%%%%%%%%
%%%%%%%%%%%%%%%%%%%%%%%%%%%%%%%%%%%%%%%%%%%%%%%%%%%%%%%%%%%%%%%%%%%%%%%%%%%%%%%
%%%%%%%%%%%%%%%%%%%%%%%%%%%%%%%%%%%%%%%%%%%%%%%%%%%%%%%%%%%%%%%%%%%%%%%%%%%%%%%

\subsection*{Title and Abstract}
\begin{frame}[fragile]{\insertsubsection}
\begin{itemize}{\small
\item Tell \LaTeX{} the \cmdbs{title} and \cmdbs{author} names in the preamble.
\item Then use \cmdbs{maketitle} in the document to actually create the title.
\item Use the \bftt{abstract} environment to make an abstract.
}\end{itemize}
\begin{minipage}{0.40\linewidth}
\inputminted[fontsize=\scriptsize,frame=single,resetmargins]{latex}%
  {structure-title.tex}
\end{minipage}
\begin{minipage}{0.55\linewidth}
\includegraphics[width=\textwidth,clip,trim=2.2in 7in 2.2in 2in]{structure-title.pdf}
\end{minipage}
\end{frame}

%%%%%%%%%%%%%%%%%%%%%%%%%%%%%%%%%%%%%%%%%%%%%%%%%%%%%%%%%%%%%%%%%%%%%%%%%%%%%%%
%%%%%%%%%%%%%%%%%%%%%%%%%%%%%%%%%%%%%%%%%%%%%%%%%%%%%%%%%%%%%%%%%%%%%%%%%%%%%%%
%%%%%%%%%%%%%%%%%%%%%%%%%%%%%%%%%%%%%%%%%%%%%%%%%%%%%%%%%%%%%%%%%%%%%%%%%%%%%%%
\subsection*{Sections}
\begin{frame}{\insertsubsection}
\begin{itemize}{\small
\item Just use \cmdbs{section} and \cmdbs{subsection} (and even \cmdbs{subsubsection}).
%\item Can you guess what \cmdbs{section*} and \cmdbs{subsection*} do?
}\end{itemize}
\begin{minipage}{0.55\linewidth}
\inputminted[fontsize=\scriptsize,frame=single,resetmargins]{latex}%
  {structure-sections.tex}
\end{minipage}
\begin{minipage}{0.35\linewidth}
\includegraphics[width=\textwidth,clip,trim=1.5in 6in 4in 1in]{structure-sections.pdf}
\end{minipage}
\begin{itemize}{\small
\item Tip: \cmdbs{tableofcontents} can automatically generate the index.
}\end{itemize}
\end{frame}

%%%%%%%%%%%%%%%%%%%%%%%%%%%%%%%%%%%%%%%%%%%%%%%%%%%%%%%%%%%%%%%%%%%%%%%%%%%%%%%
%%%%%%%%%%%%%%%%%%%%%%%%%%%%%%%%%%%%%%%%%%%%%%%%%%%%%%%%%%%%%%%%%%%%%%%%%%%%%%%
%%%%%%%%%%%%%%%%%%%%%%%%%%%%%%%%%%%%%%%%%%%%%%%%%%%%%%%%%%%%%%%%%%%%%%%%%%%%%%%
\subsection*{Labels and Cross-References}
\begin{frame}[fragile]{\insertsubsection}
\begin{itemize}{\small
\item Use \cmdbs{label} and \cmdbs{ref} to reference Sections.
\begin{itemize}
  \item This way you can reference Equations, Tables or Figures too!
\end{itemize}
%\item The \bftt{amsmath} package provides \cmdbs{eqref} for referencing
%equations.
}\end{itemize}
\begin{minipage}{0.47\linewidth}
\inputminted[fontsize=\scriptsize,frame=single,resetmargins]{latex}%
  {structure-crossref.tex}
\end{minipage}
\begin{minipage}{0.51\linewidth}
\includegraphics[width=\textwidth,clip,trim=1.8in 6in 1.6in 1in]{structure-crossref.pdf}
\end{minipage}
\end{frame}

%%%%%%%%%%%%%%%%%%%%%%%%%%%%%%%%%%%%%%%%%%%%%%%%%%%%%%%%%%%%%%%%%%%%%%%%%%%%%%%
%%%%%%%%%%%%%%%%%%%%%%%%%%%%%%%%%%%%%%%%%%%%%%%%%%%%%%%%%%%%%%%%%%%%%%%%%%%%%%%
%%%%%%%%%%%%%%%%%%%%%%%%%%%%%%%%%%%%%%%%%%%%%%%%%%%%%%%%%%%%%%%%%%%%%%%%%%%%%%%
\section{Figures and Tables}

\begin{frame}{Outline}
\tableofcontents[currentsection]
\end{frame}

%%%%%%%%%%%%%%%%%%%%%%%%%%%%%%%%%%%%%%%%%%%%%%%%%%%%%%%%%%%%%%%%%%%%%%%%%%%%%%%
%%%%%%%%%%%%%%%%%%%%%%%%%%%%%%%%%%%%%%%%%%%%%%%%%%%%%%%%%%%%%%%%%%%%%%%%%%%%%%%
%%%%%%%%%%%%%%%%%%%%%%%%%%%%%%%%%%%%%%%%%%%%%%%%%%%%%%%%%%%%%%%%%%%%%%%%%%%%%%%
\subsection*{Figures}
\begin{frame}{\insertsubsection}
\begin{itemize}
\item Add \cmdbs{usepackage\{graphicx\}} to the preamble.
\item Include an image using the \cmdbs{includegraphics} command.
%\begin{itemize}
%  \item Supported graphics formats include JPEG, PNG and PDF.
%\end{itemize}
\item \bftt{Figure} environment:
\begin{itemize}
  \item Allow \LaTeX{} to decide where the figure will go (it can ``float'').
  \item You can give the figure a caption.
  \item And also add a label and reference it with   \cmdbs{ref}.
\end{itemize}
\end{itemize}
\begin{minipage}{0.55\linewidth}
\inputminted[fontsize=\scriptsize,frame=single,resetmargins]{latex}%
  {media-graphics.tex}
\end{minipage}
\begin{minipage}{0.35\linewidth}
\includegraphics[width=\textwidth,clip,trim=2in 5in 3in 1in]{media-graphics.pdf}
\end{minipage}

%\tiny{Image license: \href{https://pixabay.com/en/animal-apple-attractive-beautiful-1239390/}{CC0}}
\end{frame}

%%%%%%%%%%%%%%%%%%%%%%%%%%%%%%%%%%%%%%%%%%%%%%%%%%%%%%%%%%%%%%%%%%%%%%%%%%%%%%%
%%%%%%%%%%%%%%%%%%%%%%%%%%%%%%%%%%%%%%%%%%%%%%%%%%%%%%%%%%%%%%%%%%%%%%%%%%%%%%%
%%%%%%%%%%%%%%%%%%%%%%%%%%%%%%%%%%%%%%%%%%%%%%%%%%%%%%%%%%%%%%%%%%%%%%%%%%%%%%%
\subsection*{Tables}
\begin{frame}[fragile]{\insertsubsection}
\begin{itemize}
\item Use the \bftt{tabular} environment from the \bftt{tabularx} package.
\item The argument specifies column alignment --- \textbf{l}eft, \textbf{r}ight, \textbf{r}ight.
\begin{exampletwouptiny}
\begin{tabular}{lrr}
Item   & Qty & Unit \euro \\
Widget & 1   & 199.99  \\
Gadget & 2   & 399.99  \\
Cable  & 3   & 19.99   \\
\end{tabular}
\end{exampletwouptiny}
\item It also specifies vertical lines; use \cmdbs{hline} for horizontal lines.
\begin{exampletwouptiny}
\begin{tabular}{|l|r|r|} \hline
Item   & Qty & Unit \euro \\\hline
Widget & 1   & 199.99  \\
Gadget & 2   & 399.99  \\
Cable  & 3   & 19.99   \\\hline
\end{tabular}
\end{exampletwouptiny}
\item Use an ampersand \keystrokebftt{\&} to separate columns.
\item Use double backslash \keystrokebftt{\bs}\keystrokebftt{\bs} to start a new row.
\end{itemize}
\end{frame}

%%%%%%%%%%%%%%%%%%%%%%%%%%%%%%%%%%%%%%%%%%%%%%%%%%%%%%%%%%%%%%%%%%%%%%%%%%%%%%%
%%%%%%%%%%%%%%%%%%%%%%%%%%%%%%%%%%%%%%%%%%%%%%%%%%%%%%%%%%%%%%%%%%%%%%%%%%%%%%%
%%%%%%%%%%%%%%%%%%%%%%%%%%%%%%%%%%%%%%%%%%%%%%%%%%%%%%%%%%%%%%%%%%%%%%%%%%%%%%%

\begin{frame}[fragile]{\insertsubsection}
\begin{itemize}
\item We can envelop a \bftt{tabular} with a \bftt{table} environment.
\begin{itemize}
   \item This allows us to float, add a caption and/or reference it later.
\end{itemize}
\end{itemize}

\begin{minipage}{0.55\linewidth}
\inputminted[fontsize=\scriptsize,frame=single,resetmargins]{latex}%
  {table-example.tex}
\end{minipage}
\begin{minipage}{0.40\linewidth}
\includegraphics[width=\textwidth,clip,trim=2in 5in 2.5in 1in]{table-example.pdf}
\end{minipage}

\end{frame}

%%%%%%%%%%%%%%%%%%%%%%%%%%%%%%%%%%%%%%%%%%%%%%%%%%%%%%%%%%%%%%%%%%%%%%%%%%%%%%%
%%%%%%%%%%%%%%%%%%%%%%%%%%%%%%%%%%%%%%%%%%%%%%%%%%%%%%%%%%%%%%%%%%%%%%%%%%%%%%%
%%%%%%%%%%%%%%%%%%%%%%%%%%%%%%%%%%%%%%%%%%%%%%%%%%%%%%%%%%%%%%%%%%%%%%%%%%%%%%%



\addtocontents{toc}{\newpage}
\section{Bibliographies}

%%%%%%%%%%%%%%%%%%%%%%%%%%%%%%%%%%%%%%%%%%%%%%%%%%%%%%%%%%%%%%%%%%%%%%%%%%%%%%%
%%%%%%%%%%%%%%%%%%%%%%%%%%%%%%%%%%%%%%%%%%%%%%%%%%%%%%%%%%%%%%%%%%%%%%%%%%%%%%%
%%%%%%%%%%%%%%%%%%%%%%%%%%%%%%%%%%%%%%%%%%%%%%%%%%%%%%%%%%%%%%%%%%%%%%%%%%%%%%%
\begin{frame}{Outline}
\tableofcontents[currentsection]
\end{frame}

%%%%%%%%%%%%%%%%%%%%%%%%%%%%%%%%%%%%%%%%%%%%%%%%%%%%%%%%%%%%%%%%%%%%%%%%%%%%%%%
%%%%%%%%%%%%%%%%%%%%%%%%%%%%%%%%%%%%%%%%%%%%%%%%%%%%%%%%%%%%%%%%%%%%%%%%%%%%%%%
%%%%%%%%%%%%%%%%%%%%%%%%%%%%%%%%%%%%%%%%%%%%%%%%%%%%%%%%%%%%%%%%%%%%%%%%%%%%%%%
\subsection*{Adding bibliography with bib\TeX}
\begin{frame}[fragile]{\insertsubsection{}}
\begin{itemize}
\item Put your references in a \bftt{.bib} file in `bibtex' database format:
\inputminted[fontsize=\scriptsize,frame=single]{latex}{bib-example.bib}
\item Most reference managers can export to bib\TeX format.
\end{itemize}
\end{frame}

%%%%%%%%%%%%%%%%%%%%%%%%%%%%%%%%%%%%%%%%%%%%%%%%%%%%%%%%%%%%%%%%%%%%%%%%%%%%%%%
%%%%%%%%%%%%%%%%%%%%%%%%%%%%%%%%%%%%%%%%%%%%%%%%%%%%%%%%%%%%%%%%%%%%%%%%%%%%%%%
%%%%%%%%%%%%%%%%%%%%%%%%%%%%%%%%%%%%%%%%%%%%%%%%%%%%%%%%%%%%%%%%%%%%%%%%%%%%%%%
\begin{frame}[fragile]{\insertsubsection{}}
\begin{itemize}
\item Each entry in the \bftt{.bib} file has a \emph{key} used to
reference it.
\item I.e., \bftt{Silvestre2012Explicit} is the key for this article:
\begin{minted}[fontsize=\small,frame=single]{latex}
@Article{Silvestre2012Explicit,
  author = {Joan Albert Silvestre-Cerda 
            and Jesus Andres-Ferrer
            and Jorge Civera},
  ...
}
\end{minted}
\item It's a good idea to use a key based on the name, year and title.
\item \LaTeX{} can automatically generate the list of references.
\item It can also automatically format your citations.
\end{itemize}
\end{frame}

%%%%%%%%%%%%%%%%%%%%%%%%%%%%%%%%%%%%%%%%%%%%%%%%%%%%%%%%%%%%%%%%%%%%%%%%%%%%%%%
%%%%%%%%%%%%%%%%%%%%%%%%%%%%%%%%%%%%%%%%%%%%%%%%%%%%%%%%%%%%%%%%%%%%%%%%%%%%%%%
%%%%%%%%%%%%%%%%%%%%%%%%%%%%%%%%%%%%%%%%%%%%%%%%%%%%%%%%%%%%%%%%%%%%%%%%%%%%%%%
\begin{frame}[fragile]{\insertsubsection{}}
\begin{itemize}
\item Use the \bftt{natbib} package with \cmdbs{citet} and \cmdbs{citep}.
\item Use \cmdbs{bibliography} to insert the references list.
\item Specify a \cmdbs{bibliographystyle}.
\end{itemize}
\begin{minipage}{0.5\linewidth}
\inputminted[fontsize=\scriptsize,frame=single,resetmargins]{latex}%
  {bib-example.tex}
\end{minipage}
\begin{minipage}{0.47\linewidth}
\includegraphics[width=\textwidth,clip,trim=1.8in 5in 1.65in 1in]{bib-example.pdf}
\end{minipage}
\end{frame}

%%%%%%%%%%%%%%%%%%%%%%%%%%%%%%%%%%%%%%%%%%%%%%%%%%%%%%%%%%%%%%%%%%%%%%%%%%%%%%%
%%%%%%%%%%%%%%%%%%%%%%%%%%%%%%%%%%%%%%%%%%%%%%%%%%%%%%%%%%%%%%%%%%%%%%%%%%%%%%%
%%%%%%%%%%%%%%%%%%%%%%%%%%%%%%%%%%%%%%%%%%%%%%%%%%%%%%%%%%%%%%%%%%%%%%%%%%%%%%%

\begin{frame}[fragile]{Mandatory exercise}

\begin{enumerate}
\item Here is the text for a short article:\footnote{Based on \url{http://www.cgd.ucar.edu/cms/agu/scientific_talk.html}}
\begin{center}
\fbox{\href{\wlnewdoc{recap-exercise.tex}}{%
Click to open this exercise in \wllogo{}}}
\end{center}
\vskip 2ex
\item Add \LaTeX{} commands to the text to make it look like this one:
\begin{center}
\fbox{\href{\fileuri/recap-exercise-solution.pdf} percent sign, \emph{escape} it with a backslash (\cmdbs{\%}).
\item To write the equation
\begin{itemize}
  \item use \cmdbs{frac\{\}\{\}} for the fraction,
  \item \cmdbs{left(} and \cmdbs{right)} for the parentheses.
\end{itemize}
\end{itemize}
\end{block}
\end{frame}

%%%%%%%%%%%%%%%%%%%%%%%%%%%%%%%%%%%%%%%%%%%%%%%%%%%%%%%%%%%%%%%%%%%%%%%%%%%%%%%
%%%%%%%%%%%%%%%%%%%%%%%%%%%%%%%%%%%%%%%%%%%%%%%%%%%%%%%%%%%%%%%%%%%%%%%%%%%%%%%
%%%%%%%%%%%%%%%%%%%%%%%%%%%%%%%%%%%%%%%%%%%%%%%%%%%%%%%%%%%%%%%%%%%%%%%%%%%%%%%

\section{What's Next?}

%%%%%%%%%%%%%%%%%%%%%%%%%%%%%%%%%%%%%%%%%%%%%%%%%%%%%%%%%%%%%%%%%%%%%%%%%%%%%%%
%%%%%%%%%%%%%%%%%%%%%%%%%%%%%%%%%%%%%%%%%%%%%%%%%%%%%%%%%%%%%%%%%%%%%%%%%%%%%%%
%%%%%%%%%%%%%%%%%%%%%%%%%%%%%%%%%%%%%%%%%%%%%%%%%%%%%%%%%%%%%%%%%%%%%%%%%%%%%%%
\begin{frame}{Outline}
\tableofcontents[currentsection]
\end{frame}

\subsection*{Document templates}
\begin{frame}{\insertsubsection}
\begin{itemize}
  \item TFG/TFM template using \cmdbs{documentclass\{book\}}:

\begin{center}
\fbox{\href{\wlnewdoc{template_TFG.tex}}{%
Click to open this template in \wllogo{}}}
\end{center}

  \item Slides template using \bftt{beamer}:

\begin{center}
\fbox{\href{\wlnewdoc{template_slides.tex}}{%
Click to open this template in \wllogo{}}}
\end{center}

\item Many other document templates at \wllogo{}:
\begin{itemize}
  \item \url{https://www.overleaf.com/latex/templates}
\end{itemize}

\end{itemize}
\end{frame}

%%%%%%%%%%%%%%%%%%%%%%%%%%%%%%%%%%%%%%%%%%%%%%%%%%%%%%%%%%%%%%%%%%%%%%%%%%%%%%%
%%%%%%%%%%%%%%%%%%%%%%%%%%%%%%%%%%%%%%%%%%%%%%%%%%%%%%%%%%%%%%%%%%%%%%%%%%%%%%%
%%%%%%%%%%%%%%%%%%%%%%%%%%%%%%%%%%%%%%%%%%%%%%%%%%%%%%%%%%%%%%%%%%%%%%%%%%%%%%%
\subsection*{Installing \LaTeX{}}
\begin{frame}{\insertsubsection}
\begin{itemize}
\item Overleaf is a cool on-line web \LaTeX{} editor.
\item To run \LaTeX{} off-line on your own computer, you need to install a \LaTeX{}
\emph{distribution}. 
\item A distribution includes a \bftt{latex} program
and (typically) several thousand packages.
\begin{itemize}
\item On Windows: \href{http://miktex.org/}{Mik\TeX} or \href{http://tug.org/texlive/}{\TeX Live}
\item On Linux: \href{http://tug.org/texlive/}{\TeX Live}
\item On Mac: \href{http://tug.org/mactex/}{Mac\TeX}
\end{itemize}
\item You'll also want a \href{http://en.wikipedia.org/wiki/Comparison_of_TeX_editors}{text editor with \LaTeX{} support}. 
\begin{itemize}
  \item We recomend \href{https://www.lyx.org/}{LyX} or \href{https://kile.sourceforge.io/}{Kile}
\end{itemize}
\end{itemize}
\end{frame}

%%%%%%%%%%%%%%%%%%%%%%%%%%%%%%%%%%%%%%%%%%%%%%%%%%%%%%%%%%%%%%%%%%%%%%%%%%%%%%%
%%%%%%%%%%%%%%%%%%%%%%%%%%%%%%%%%%%%%%%%%%%%%%%%%%%%%%%%%%%%%%%%%%%%%%%%%%%%%%%
%%%%%%%%%%%%%%%%%%%%%%%%%%%%%%%%%%%%%%%%%%%%%%%%%%%%%%%%%%%%%%%%%%%%%%%%%%%%%%%
\subsection*{Online Resources}
\begin{frame}{\insertsubsection}
\begin{itemize}
\item \href{http://en.wikibooks.org/wiki/LaTeX}{The \LaTeX{} Wikibook} 
\begin{itemize}
  \item Excellent tutorials and reference material.
\end{itemize}
\item \href{http://tex.stackexchange.com/}{\TeX{} Stack Exchange} 
\begin{itemize}
  \item Ask questions and get excellent answers quickly.
\end{itemize}
\item \href{http://www.latex-community.org/}{\LaTeX{} Community}
\begin{itemize}
  \item A large online forum.
\end{itemize}
\item \href{http://ctan.org/}{Comprehensive \TeX{} Archive Network (CTAN)} 
\begin{itemize}
  \item Over four thousand packages plus documentation.
\end{itemize}
\item Google will usually get you to one of the above.
\end{itemize}
\end{frame}


\end{document}
